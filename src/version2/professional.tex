% mainfile: ./cv.tex
\begin{itemize}
    \item {\bf Graduate Research Assistant} \hfill Spring 2012 --- Present\\
        \textit{Discovery Analytics Center, Virginia Tech} \hfill \textit{Arlington, VA}
      \vspace{-0.5em}
      \begin{itemize}
         \resitem{\textbf{Time Series Forecasting
          for Cyber-Physical Systems:} 
          \begin{itemize}[nosep]
              \item Investigated utility of Sequence-to-Sequence (Seq2Seq) models for
          multi-variate time series state forecasting
          \item Collaboratively developed Seq2Seq architecture suitable for time-series forecasting and showed 18.83\% improvement in load forecasting. Published in IJCAI'16
          \end{itemize}
          
         }
          \resitem{\textbf{Human-aware Approaches for Event Encoding:}
          \begin{itemize}[nosep]
              \item Built Hybrid Human-AI system for
          encoding political events from news media
              \item Developed ML system to characterize uncertainty for deciding which documents and tasks require supervision
              \item Showcased performance improvement of 14\% in Precision, 13.7\% in Recall and  12.5\% in Quality Score
          \end{itemize} 
          }

          \resitem{\textbf{EMBERS (IARPA Open Source Indicators program):}
          \begin{itemize}[nosep]
              \item Worked as a part of multi-disciplinary multi-university team on
          building a real-time forecasting system of spatio-temporal events like
          Civil Unrest 
              \item Led development of model to identify planned
          events from social media, news and other sources and
          approaches to create automated narrative (sequence of) stories
          from real-time news
              \item Collaborated in automated geo-coding and model fusion research. Also contributed to overall system architecture and ingestion/enrichment of data sources
            \item EMBERS was sponsored by a three-year contract for approx. \$13.36M from the IARPA OSI Program
        \end{itemize}
          }

          \resitem{\textbf{SAFE (IARPA Mercury):}
          \begin{itemize}[nosep]
              \item Directed research efforts for development of
          fusion system responsible for accepting
          alerts (tuple of <forecast date,location,actors,type>)
          from all underlying models and deciding which alerts
          to merge or suppress
          \item  Built models
          to perform real-time clustering of alerts and to predict
          expected quality of an incoming alert which can then be used
          to merge/suppress alert
          \end{itemize}
          }
          \resitem{\textbf{IGACAT (IARPA Functional Genomic and
          Computational Assessment of Threats Program):}
          \begin{itemize}[nosep]
              \item Built DL model based on (CNN, LSTM) and Attention for performing few shot
          learning to identify if a given nucleotide sequence is a toxin. Work published as poster paper at Machine Learning and Computation Biology conference 2019 (MLCB) and as abstract at ASM BioThreats
              \item Researching on Heterogenous interaction based knowledge graphs (KG) for multiple tasks such as identification of genetic functions, possibility of threat, taxonomy etc
          \end{itemize}
          }

      \end{itemize}

\item {\bf Intern, advised by Dr.\ Prashanta Panigrahi} \hfill Nov 2008 --- Jan 2009 \\
  \textit{Indian Institute of Scientific Education and Research}\hfill \textit{Kolkata, India}
  
  \begin{itemize}[nosep] \resitem{Studied the feasibility of a quantum computing system using common and attainable finite dimensional multipartite quantum states}
  \end{itemize}

\item {\bf Intern, mentored by Dr.\ Biswajit Chakraborty }\hfill May-June 2010 \\
    \textit{National Institute of Oceanography}\hfill \textit{Goa, India}

  \begin{itemize}[nosep]\resitem{Estimated geophysical parameters using SONAR Backscatter data obtained off an experiment on the Western Continental shelf of India. Developed processing and cleansing
      techniques to process back-scatter data and used simplex method for estimation}
  \end{itemize}
%\item
\end{itemize}
