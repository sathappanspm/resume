% mainfile: ./cv.tex
\begin{itemize}
    \item {\bf Virginia Tech} \hfill Spring 2012 --- Present\\
        \textit{GRA, Discovery Analytics Center} \hfill \textit{Arlington, VA}
  %\ressubheading{Virginia Tech}{Arlington, VA}{GRA, Discovery Analytics 
      %Center}{Spring 2012 --- Present }\vspace{0.5em}
      \begin{itemize}
 
          \resitem{\textbf{Time Series Anomaly Detection and Forecasting
          for Cyber-Physical Systems:} Studied utility of
          Sequence-to-Sequence (Seq2Seq) models for the task of
          multi-variate time series state forecasting. Characterized
          need for changes to traditional seq2seq architecture
          (developed predominantly for NLP) for applying in time-series
          and thereby collaborated in effort to build dynamic attention
          networks (DyAT) for state forecasting.The work is currently
          under review in IJCAI 2019}

          \resitem{\textbf{Automated Approaches for Event Encoding:}
          The project is aimed building a cost effective system for
          encoding political events like Protests, Military Actions etc
          from news media. The major goal of the system is to create a
          system with better understanding of uncertainty i.e., it knows
          when it doesnt know. Characterizing uncertainty helps
          determine which sections/documents need human supervision.}

          \resitem{\textbf{IGACAT (IARPA Functional Genomic and
          Computational Assessment of Threats Program):}
          Researching on methods to characterize threats from specific
          genes/proteins using heterogenous interaction graph created
          from pubmed/medline abstracts and information of GO functions,
          sequence groups and cluster information obtained from
          knowledge bases like UNIPROT, TREMBL etc.
          Built deep learning models (CNN, LSTM) for performing few shot
          learning to identify functions of a nucleotide sequence by
          embedding both the sequence and label (the functions) in the
          same space.}
          
          \resitem{\textbf{SAFE (IARPA Mercury):} Led development of
          fusion system. The fusion system is responsible for accepting
          alerts from all underlying models and deciding which alerts
          (each alert is a tuple of <forecast date, location, actors,
          event-type>) to merge and which ones to suppress. Built models
          to perform real-time clustering of alerts and to predict
          expected quality of an incoming alert which can then be used
          to merge/suppress alerts.}

          \resitem{\textbf{EMBERS (IARPA Open Source Indicators program):}
          Worked as a part of multi-disciplinary multi-university team on
          building a real-time forecasting system of spatio-temporal events like
          Civil Unrest. EMBERS was sponsored by a 
          three-year contract for approx. \$13.36M from the IARPA OSI Program.
          Led development of model to identify planned
          events from social media, news and other sources and
          approaches to create automated narrative (sequence of) stories
          from real-time news. Also
          researched on approaches to
          perform automated geo-coding (geo-disambiguation) of news and
          other textual content. Besides contributed in system
          architecture and ingestion/enrichment of data sources.}\vspace{0.5em}

      \end{itemize}
          
\item {\bf Indian Institute of Scientific Education and Research} \hfill Nov 2008 --- Jan 2009 \\
  \textit{Intern, advised by Dr.\ Prashanta Panigrahi}\hfill \textit{Kolkata, India}
  
  \begin{itemize} \resitem{Studied the feasibility of a quantum computing system using common and attainable finite dimensional multipartite quantum states.}
  \end{itemize}\vspace{0.5em}

\item {\bf National Institute of Oceanography}\hfill May-June 2010 \\
    \textit{Intern, mentored by Dr.\ Biswajit Chakraborty}\hfill \textit{Goa, India}

  \begin{itemize}\resitem{Estimated geophysical parameters using SONAR Backscatter data obtained off an experiment on the Western Continental shelf of India. Developed processing and cleansing
      techniques to process backscatter data and used simplex method for estimation.}
  \end{itemize}\vspace{0.5em}
%\item
\end{itemize}
